%!TEX root = ../thesis.tex

\section{背景}
近年,機械学習を用いた自律移動に関しての研究が盛んに行われている.本研究室でも,機械学習を用いた画像に基づく人追従行動の生成に関する研究を行ってきた.

パシンら\cite{pasin1}\cite{pasin2}\cite{pasin3}は,人追従行動の生成に強化学習を用いている.

岡田ら\cite{okada}はこれらの技術を応用し,カメラ画像に基づく人追従行動を獲得している.ここでの教師信号はカメラ画像とルールベース制御器の出力である.

  \begin{figure}[h]
    \centering
    \begin{minipage}[c]{100mm} 
        \centering
        \includegraphics[width=100mm]{images/okada_learning_phase_system.png}
        \subcaption{Learning phase}
    \end{minipage} \\
    \vspace{1em} % 画像とキャプションの間にスペースを追加
    \begin{minipage}[c]{100mm} 
        \centering
        \includegraphics[width=100mm]{images/okada_following_phase_system.png}
        \subcaption{Following phase}
    \end{minipage}
    \caption{The proposed method for learning of the person-following behavior\cite{okada}}
    \label{Fig:okada_system}
  \end{figure}
% \subsection{etc...}
% \subsubsection{etc...}

\newpage
