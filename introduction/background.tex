%!TEX root = ../thesis.tex

\section{背景}
  近年,機械学習を活用した自律移動に関する研究が盛んに行われている.本研究室でも,機械学習を用いて画像に基づく人追従行動の生成に関する研究を行ってきた.
  パシンら\cite{pasin1}\cite{pasin2}\cite{pasin3}は,引き紐を利用して画像に基づく人追従行動を生成する手法を提案している.この手法では,深層強化学習\cite{hado}を活用しており,引き紐に取り付けられたポテンショメータでリンクの角度を取得し,それに応じた報酬をエージェント(ロボット)に与えて強化学習\cite{leslie}する.そして,画像と行動を深層学習\cite{yann2}することで,画像に基づいて人追従する行動を生成できることを示した.
  \figref{Fig:pasin_system}にシステムの概要を示す.入力は画像で,出力は直進,左旋回,右旋回のいずれかの行動である.引き紐は,報酬を与えるためのみに利用され,ロボットの行動選択の入力としては用いていない.目的とする行動は人の追従であるため,選択された行動が適切であれば紐の向きとロボットの進行方向が一致するが,適切でない場合は一致しない.ここで,紐の向きと進行方向が一致しているときに高い報酬を与えている.

  さまざまな行動と画像に対して,リンクの角度に応じた報酬を与えることで,徐々に人を追従する行動を選択していった.学習時間は約20分であるが,強化学習の特性により行動がランダムに選択されていた.この際に,ロボットは望まない行動を繰り返すため,追従対象者に比較的負担がかかるという問題があった.

\newpage

  \begin{figure}[h]
    \centering
    \includegraphics[keepaspectratio, scale=1.25] {images/pdf/pasin_system}
    \caption[Proposed method]{Proposed method (source: \cite{pasin1})}
    \label{Fig:pasin_system}
  \end{figure}

  \vspace{0.5cm}

  岡田ら\cite{okada}は,強化学習を使用せず,深層学習により画像に基づく人追従行動を生成する手法を提案している.この手法は,後述するBojarskiら\cite{bojarski}の技術(end-to-end学習)を人追従問題に応用しており,強化学習を使用していないため,学習時のロボットの行動がランダムに選択されることはない.また,学習時はルールベース制御器でロボットを制御しているので,常に人を追従する.
  \figref{Fig:okada_system}にシステムの概要を示す.まず,学習時には,追従対象者が引き紐を操作する.引き紐にはパシンらと同様にポテンショメータが取り付けられており,ヨ―関節の変位角が0度となるようにロボットは直進や左旋回,右旋回のいずれかの行動で制御される.並行して,これらの行動とカメラ画像を深層学習器にオンラインでend-to-end学習する.学習後は,追従対象者が引き紐を操作しなくても,深層学習器によりカメラ画像を入力するだけで,出力は直進や左旋回,右旋回といった行動を選択する.つまり,学習時のルールベース制御器(引き紐による人追従行動)を模倣するような深層学習器(カメラ画像による人追従行動)になっている.

  これまで本研究室では,引き紐を用いて人追従行動を生成してきたが,測域センサの反射強度を利用することで同様の人追従行動を生成できる可能性がある.これにより,学習時のセンサの使用
  に新たな選択肢が加わる.

\newpage

\vspace{2cm}

  \begin{figure}[h]
    \centering
    \begin{minipage}[c]{120mm} 
        \centering
        \includegraphics[width=110mm]{images/pdf/okada_learning_phase_system}
        \subcaption{Learning phase}
    \end{minipage} \\
    \vspace{1em} % 画像とキャプションの間にスペースを追加
    \begin{minipage}[c]{120mm} 
        \centering
        \includegraphics[width=110mm]{images/pdf/okada_following_phase_system}
        \subcaption{Following phase}
    \end{minipage}
    \caption[The proposed method for learning of the person-following behavior]{The proposed method for learning of the person-following behavior (source: \cite{okada})}
    \label{Fig:okada_system}
  \end{figure}

\newpage

% \subsection{etc...}
% \subsubsection{etc...}