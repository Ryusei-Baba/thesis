%!TEX root = ../thesis.tex
\chapter*{概要}
\thispagestyle{empty}
%
\begin{center}
  \scalebox{1.5}{ラグランジュ法}\\
\end{center}
\vspace{1.0zh}
%
 三次元空間で運動する場合,並進運動と回転運動の 2 つの成分に分けられる.並進運動
はニュートンの運動方程式,回転運動はオイラーの運動方程式で記述することができ,こ
れらを合わせてニュートン・オイラー法と呼ぶ.ニュートン・オイラー法を利用すること
で,各リンクの重心に作用する力とトルクを計算することができる.この代替手法とし
て,ラグランジュ法が挙げられる.ラグランジュ法は,エネルギベースのアプローチで,
剛体リンクを備えた直動マニピュレータにある程度特化している.
\vspace{1.0zh}

キーワード: ラグランジュ,運動エネルギ,位置エネルギ
