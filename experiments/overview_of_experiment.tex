%!TEX root = ../thesis.tex

\section{実験の概要}

  本研究では,2DLiDARの反射強度を利用したルールベース制御器を用いた人追従行動を,カメラ画像で模倣学習することを課題としているが,ルールベース制御器において,最大反射強度の方にロボットが追従する手法となっている.そのため,実験で使用する再帰反射テープよりも高い反射強度を取得してしまうと人追従行動を継続できない恐れがある.そこで,2DLiDARの反射強度の実験と提案手法による人追従の実験の2つに分けて行う.それぞれの実験の概要を以下に示す.

  \subsubsection*{<実験1:2DLiDARの反射強度の実験>}
  \begin{itemize}
    \item 壁の反射強度を計測する実験
    \item 再帰反射テープの反射強度を計測する実験
    \item 学習フェーズで使用する環境を計測する実験
    % \item ルールベース制御器を用いた人追従の実験
  \end{itemize}
  
\newpage

  \subsubsection*{<実験2:提案手法による人追従の実験>}
  ルールベース制御器により10個の学習モデルを作成し,学習モデル一つ一つに対してテストを行う.つまり,実験を10回行い,提案手法の有効性の検証をする.

  \vspace{1cm}

  実験環境は,\figref{Fig:RobotGuidance_cit3f}に示すように千葉工業大学津田沼キャンパス2号館3階の廊下とした.実験は天候による影響を少なくするため,夜間に実験を行った.また,服装による影響を少なくするため,追従対象者は\figref{Fig:Sequence of the experiment}に示すような青いビブスを着用し,学習フェーズと追従フェーズに分けて実験を行った.

  \begin{figure}[h]
    \centering
    \includegraphics[keepaspectratio, scale=0.70] {images/RobotGuidance_cit3f.png}
    \captionsetup{justification=raggedright} % キャプションを左寄せに
    \caption{The environment of the experiment}
    \label{Fig:RobotGuidance_cit3f}
  \end{figure}

  \begin{figure}[h]
    \centering
    \begin{minipage}[c]{65mm} 
        \centering
        \includegraphics[height=40mm]{images/figure.png}
        \subcaption{Learning phase}
    \end{minipage}
    \begin{minipage}[c]{65mm} 
        \centering
        \includegraphics[height=40mm]{images/figure.png}
        \subcaption{Following phase}
    \end{minipage}
    \caption{Sequence of the experiment}
    \label{Fig:Sequence of the experiment}
  \end{figure}

\newpage
