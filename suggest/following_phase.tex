%!TEX root = ../thesis.tex

\section{追従フェーズ}

  追従フェーズの概要を\figref{Fig:RobotGuidance_following_system}に示す.追従フェーズでは,学習フェーズで学習したモデルを用いる.ここでは,2DLiDARを使用せず,深層学習器の出力がロボットの行動になる.つまり,2DLiDARの反射強度を利用したルールベース制御器の出力がロボットの行動になるのではなく,画像を入力とした深層学習器の出力がロボットの行動になる.

  \begin{figure}[h]
    \centering
    \includegraphics[keepaspectratio, scale=0.45] {images/RobotGuidance_following_system.png}
    \captionsetup{justification=raggedright} % キャプションを左寄せに
    \caption{Proposed method in the following phase}
    \label{Fig:RobotGuidance_following_system}
  \end{figure}

  \begin{figure}[h]
    \centering
    \includegraphics[keepaspectratio, scale=0.55] {images/RobotGuidance_following_phase_leg.png}
    \captionsetup{justification=raggedright} % キャプションを左寄せに
    \caption{Without retroreflective tape}
    \label{Fig:RobotGuidance_following_phase_leg}
  \end{figure}

\newpage
