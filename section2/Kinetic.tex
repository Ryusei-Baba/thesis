%!TEX root = ../thesis.tex

 \section{運動エネルギ}
% \subsection{RoboCup}

% \begin{figure}[hbtp]
%   \centering
%  \includegraphics[keepaspectratio, scale=0.8]
%       {images/mani.png}
%  \caption{Manipulator}
%  \label{Fig:manipulator}
% \end{figure}
マニピュレータの運動エネルギの式を定義する.$i$ 番目のリンクの運動エネルギ $k$ は,
以下の式(2.1)のように表すことができる.
\begin{equation}
     k_i= \frac{1}{2} m_i v_{Ci}^{T} v_{Ci} +\frac{1}{2} \; {}^{i}{\omega}_{i}^{T} \; {}^{Ci} I_i\; {}^{i}{\omega}_i
\end{equation}
ここで,式(2.1)の第 1 項はリンクの重心の速度による運動エネルギであり,第 2 項はリンク
の角速度による運動エネルギである.マニピュレータの総運動エネルギは,個々のリンク
の運動エネルギの合計なので,
\begin{equation}
     k= \sum^{n}_{i=1}k_{i}
\end{equation}
と表すことができる.式(2.1)の $v_{Ci}$ と ${}^{i}{\omega}_{i}$ は$\theta$と$\dot{\theta}$の関数なので, マニピュレータの運動エ
ネルギは関節の位置と速度 $k(\theta,\dot{\theta})$ の関数としてスカラで記述できる.実際,マニピュレー
タの運動エネルギは次のように与えられる.
\begin{equation}
     k(\theta,\dot{\theta})= \frac{1}{2} \dot{\theta}^T M(\theta) \dot{\theta}
\end{equation}
ここで,$M(\theta)$は $n×n$ で表されるマニピュレータの質量行列である.式(2.3)の形式は,二
次関数として知られている.展開すると,結果として得られるスカラ方程式が二次関数に
依存する項のみで構成される.さらに,総運動エネルギは常に正でなければならないた
め,マニピュレータの質量行列は正定行列である必要がある.正定行列は,二次関数が常
に正のスカラであるという特性を持つ行列である.
\newpage
