%!TEX root = ../thesis.tex

%  \section{運動エネルギ}
% \subsection{RoboCup}

% \begin{figure}[hbtp]
%   \centering
%  \includegraphics[keepaspectratio, scale=0.8]
%       {images/mani.png}
%  \caption{Manipulator}
%  \label{Fig:manipulator}
% \end{figure}
三次元空間で運動する場合,並進運動と回転運動の 2 つの成分に分けられる.並進運動
はニュートンの運動方程式,回転運動はオイラーの運動方程式で記述することができ,こ
れらを合わせてニュートン・オイラー法と呼ぶ.ニュートン・オイラー法を利用すること
で,各リンクの重心に作用する力とトルクを計算することができる.この代替手法とし
て,\textbf{ラグランジュ法}が挙げられる.ラグランジュ法は,エネルギベースのアプローチで,
剛体リンクを備えた直動マニピュレータにある程度特化している.
\newpage
