%!TEX root = ../thesis.tex

 \section{位置エネルギ}
% \subsection{RoboCup}

% \begin{figure}[hbtp]
%   \centering
%  \includegraphics[keepaspectratio, scale=0.8]
%       {images/mani.png}
%  \caption{Manipulator}
%  \label{Fig:manipulator}
% \end{figure}

マニピュレータの位置エネルギの式を定義する.$i$ 番目のリンクの位置エネルギ $u_i$ は,
以下の式(2.4)のように表すことができる.
\begin{equation}
     u_i =-m_i {}^{0}g^T \; {}^{0}P_{Ci} + u_{refi}
\end{equation}
ここで, ${}^{0}g$ は $3 × 1$ の重力ベクトル,${}^{0}P_{Ci}$ は $i$ 番目のリンクの重心位置を示すベクト
ル,$u_{refi}$ は $u_i$ の最小値が得られるように選択された定数である.マニピュレータに保存
される位置エネルギの合計は,個々のリンクの位置エネルギの合計なので,
\begin{equation}
     u= \sum^{n}_{i=1}u_{i}
\end{equation}
と表すことができる.式(2.4)の ${}^{0}P_{Ci}$は$\theta$の関数なので,マニピュレータの位置エネルギは関
節位置$u(\theta)$の関数としてスカラで記述できる.
\newpage
