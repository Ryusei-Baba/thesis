%!TEX root = ../thesis.tex

\section{学習フェーズ}

  学習フェーズで使用するルールベース制御器を\figref{Fig:okada_rule-based_contoroller}に示す.ルールベース制御器は,引き紐に取り付けられたポテンショメータからのリンクの角度を入力とし,ルールに従いロボットは行動を選択する.ルールベース制御器からの出力を\tabref{tab:actions_control_parameters}に示す.なお,並進速度は0.1 \,[m/s]で一定である.

  \begin{figure}[h]
    \centering
    \includegraphics[keepaspectratio, scale=1.1] {images/pdf/okada_rule-based_contoroller}
    \caption[Action according to angles of the joint]{Action according to angles of the joint (source: \cite{okada})}
    \label{Fig:okada_rule-based_contoroller}
  \end{figure}

  \vspace{1cm}

  \begin{table}[ht]
    \caption{The control rule and action parameters}
    \label{tab:actions_control_parameters}
    \centering
    \begin{tabular}{cccc}
    \hline
    Action & Control rule & Linear velocity & Angular velocity \\ 
    \hline
    \hline
    Turn Left & $\theta_{\mathrm{yaw}} > 10 \, \mathrm{deg}$ & 0.1 \,m/s & 0.2 \,rad/s \\ 
    Forward & $-10 \, \mathrm{deg} < \theta_{\mathrm{yaw}} < 10 \, \mathrm{deg}$ & 0.1 \,m/s & 0 \,rad/s \\ 
    Turn Right & $\theta_{\mathrm{yaw}} < -10 \, \mathrm{deg}$ & 0.1 \,m/s & -0.2 \,rad/s \\ 
    \hline
    \end{tabular}
    \end{table}
  
\newpage